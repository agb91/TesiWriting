% Chapter 3

\chapter{General organization of the project} % Main chapter title

\label{Chapter3} % For referencing the chapter elsewhere, use \ref{Chapter1} 

%----------------------------------------------------------------------------------------

In this chapter some information is exposed about the main actors and the general organization of the project. 

\section{The roles of the designer and of the users}
Firstly we explain how the production of gAn Web is organized:
In the gAn Web design process the interaction between me and users is very strict: in fact for some months, divided in blocks of some weeks, I have worked together directly with the users in their standard work environment, both during the development process and after, to test the result. 

A fundamental point to understand is that the difficulties of this project are not only related to the technical part: the most difficult part was to understand how an ideal interaction takes place, what are the needs of the users, how can the application be really helpful for them.
 
It is interesting to note that the domain in which this application works is quite complex and very different from the domains in which the developer was specialized, therefore working directly with the users is necessary to let the developer understand better how the domain works.
This allowed the designer to understand better how exactly the work processes take place, and what are the needs of the users. 
It is important to underline that in this case the users are quite particular: their education level is very high (almost all of them have a Phd), and their needs are very specialized (related with the physics research world).

In this design process we can identify 3 main actors: 
\begin{enumerate}

\item
The developer, who produced the web interface (gAn Web), and wrote this document.

\item 
Two super users (university professors in Brescia), that are also co-developer of the application behind the interface (gAn). We call them "super-users" because they cover more than one role in the development of the application. In fact, they act three roles: the role of the users (in particular they checked the application at every stage, and also before the general tests with the users, so they can be considered pilot-users), the role of the supervisors of the global project, and the role of co-developer.
They are physicist so they can play a "jolly role" creating an important bridge between the physic domain and the software engineering domain. 

\item A group of generic users (around 20) at CERN. This main group is a resource for the final testing of the application, that is designed on their needs.
 
\end{enumerate}

We remark that in this project the user is not only a customer: the user is participatory. In particular the two super-users are consulted to take every important decision in the project, so they can be denoted as co-developer. Also the generic users have an important part: the tests with the users reveal problems, and some users are source of ideas, both during the production of the application and after, during the normal use of gAn Web.

\section{General organization of the design process }
According to the best practices of the Human-Computers Interaction science this project follows the star-shaped life cycle: the development is a group of activities that aren't carried on just one time but iteratively in all the life of the project. The most important activity is the evaluation of the product under the usability and the feasibility points of view. Around this central activity there are activities related to find the real needs of the users, select the functionalities to implement, prototyping, development, test. All the activities are repeated more times, as explained in the following section.    

\section{The stages of the design project}

The design process can be divided in three stages:

\begin{enumerate}

% 1
\item An early stage, simpler, just to investigate what are the best ways to implement the required functionalities that are the really important ones and to test with a little group of super-users (two) if the software really coincides with requirements; The goal of this stage was to give an initial direction to the development and to improve the developer's knowledge of the domain. At the end of this stage, the comparison between the developer and the pilot users helped to correct the way.

% 2
\item An intermediate stage, more complex, with advanced functionalities obtained listening the request of the super users.  
The designer needed at this point to execute some tests with the maximum possible number of users to understand if the product was acceptable and useful (and how to improve it). 

In this stage the tests with the users were divided into two blocks:  
The first one was another test with the pilot users, their hints have been applied before let the group of generic users work with the application;
the second block is a test with a large group of users who worked with the application in a real work situation.

The tests with the users showed a lot of useful information and allowed the developer to find a lot of problems: the solutions to these problems gave the birth at the final version. 

% 3
\item
The third stage is the final stage. At this point the application is modified according to the observations of the users (we remark here that the direct impact of the users leads to a lot of modifications). It is important to notice that the third stage (the last one) is a potentially everlasting stage: the needs of the users are permanently changing and evolving, new ideas and proposals still come from them, the application is designed to be adaptable, and to try to satisfy the unknown needs of future users, and unavoidably it needs continuous upgrades and modifications. 

\end{enumerate}


\begin{figure}[H]
\centering
\includegraphics[scale=0.5]{TimeLine.png} 
\caption{An overview of the timeline of the project}
\end{figure}


\section{The role of aesthetics} 
In this project the role of aesthetics is considered partially in the intermediate version and fully taken into account in the final version, immediately before tests this application with the final users.
In the first and intermediate versions we can observe some overlapping surfaces, some asymmetrical figures, and some others aesthetic problems, because in the design process we mainly focus on the functionalities, but the aesthetics plays an important role in the satisfaction of the users and it is carefully considered in the third version of the interface.  

\section{Analysis with inspection methods and tests with users}
These are methods to try to judge the application (if necessary more than once) during the development process. 
The goals of these evaluations are to understand if gAn Web meets the requirements, if it has a good level of usability, if the users can work correctly and with satisfaction with it, and finally if the application respects the general principles of the human-computer interaction (for example, the ten principles proposed by Nielsen [https://www.nngroup.com/articles/
ten-usability-heuristics/][6] that are the base of the human-machine interaction science).
There are two different approaches:

\begin{enumerate}

% 1
\item For what concerns the inspection analysis (so the tests that don't involve directly the users but are done by specialized evaluators, who check the application against some principles) we note that the evaluator was only one, the developer. This can be a big limit because only one evaluator can find only a little part of the problems. Also, the assessor was the same person who designed the interface; given that this kind of analysis will not be the best source of feedbacks about the application a good solution is to check the various parts of the application against the Nielsen principles ( [\url{https://www.nngroup.com/articles/ten-usability-heuristics/}][6] ) in the meanwhile these parts are designed and created.

%2
\item For what concerns the opportunity to test the application with the users we have a good amount of users, they are very accessible, their roles and characteristics are known and precisely defined, and we also have two super-users that can help to drive the design process towards the right direction (actually these users play a role that can be considered also a designer-customer role). At this point we can hope that the comparison between the users is the best available source for new ideas and a way to find and understand problems.    

\end{enumerate}

We note that in this project the evaluation is mostly a "formative evaluation" (a test that takes place during the design process, and can influence the design process itself), and only in the last stage it becomes a "summary evaluation" (an evaluation done at the end of the work to judge the result). This is because the comparison between the users is the only intelligent way to understand in which direction development should be performed and to enter in their "domain". So the formative evaluation, both from the two super-users, and from the generic users, is probably the most important source of requirements and solutions.   

\section{Prototyping and mock-up}
In Human Computer interaction a "mock-up" is a prototype of an application, a version of the application with incomplete functionalities. There is a wide range of mock-ups, from the basic paper based ones, useful at the beginning of the project, to the interactive ones, that are very close to the final application. They are useful to test different solutions with the users.
In the testing of gAn Web the use of mock-ups is an example of interactive mock-ups. All the mock-ups are produced directly with HTML-SCSS-javascript-PHP so they are real part of the website. In that cases they are considered as mock-ups only because some functionalities related to gAn on the back-end are still not implemented at the moment of the creation of the front-end. In the final version almost all the website is real and working. Does it mean that the application is ready and finished? No, it doesn't: still the server is not working properly (we are pretty sure that we have to format it and re-configuring some applications installed on that server) and only some analysis prepared for the software are ready (the output of the others is actually a mock-up). The decision of working with interactive mock-ups directly produced by web programming technologies is related to the fact that the developer is quite inexperienced using mock-up software but quite used to work with web development, so working directly with the web development seems to be a quite fast way, and the interactive mock-ups are perfect to be used to show the progresses of the application to the super-users.
Often the problem of interactive mock-ups are that the developer is reluctant to modify them. and it is true, but in this case the continuous modification of the application through the different development stages is very important to reach progressively to a good solution able to meet the users needs. In fact we can see that the system in the different stages is very different, actually at each iteration the production of the website re-started almost from zero. 

\section{Ambiguities and doubts}
Every time in the design process there is a doubt related to a requirement the adopted procedure is to ask directly to the users what solution to implement. This means asking before to the super users (for motivations related to their role of supervisors of this project) and possibly observe the behavior of general 
users during the tests.

\section{How the various stages are presented}
According to this division in three main stages we can organize the following chapters in this way:
each chapter is related to a stage of the development: early stage (chapter  \ref{Chapter3} ), intermediate stage (chapter \ref{Chapter4} ) and final stage (chapter \ref{Chapter5} ). Tests with the users, strictly related to chapter 5 are presented and discussed in chapter 6 ( \ref{Chapter6} ).
For each of the three stages of the development this document will explain:

\begin {enumerate}

\item
The requirements of the web interface.

\item
The functional analysis of the requirements through some expected scenarios

\item
The resulting prototype (and implementation)  

\end {enumerate} 



 