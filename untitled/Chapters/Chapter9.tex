% Chapter 9

\chapter{Web Development} % Main chapter title

\label{Chapter 9} % For referencing the chapter elsewhere, use \ref{Chapter1} 

%----------------------------------------------------------------------------------------

This chapter aims to explain what choices were made and why from the web development point of view. It is interesting to analyse which are the particular requirements of this software and what solutions were chosen to satisfy them. 

\section{Requirements from the Web Development Point of View}

From the Web Development's point of view the first step to do is to identify what are the most important goal to achieve. They are uncorrelated but not conflict with the requirements already analyzed related with the Human Machine Interaction's point of view. Following there are some observations:

\begin{enumerate}

% 1
\item
The first thing that it is important to guarantee is a strong division between front end and back end. This is because both part are important, but the way in which they work are very independent, and they can evolve in different ways.
As explained in the previous chapters the front-end, so the way in which the information is presented to the user is a particularly sensitive issue that needed a special attention and a deep analysis related to the interaction; The back-end is also important because needs to work on particular topics related to the communication with ROOT Framework, the security, the persistence of data. An important observation is that the front-end changes in the time, according to the new features and in reaction of the observations of the users, but the back-end changes faster. The challenge of the front-end is the needs to remain more stable and be re-usable when the logic behind, the back-end changes. For example: almost twice a month the algorithms of the existing analysis of gAn change, and also new analysis are added (sometimes some old ones are removed). The output of the back-end changes rapidly, but almost always the front-end, that is flexible can manage automatically the situation showing the results to the users in an organized way. Also the drivers able to trigger the changes are different: the users for the front-end, the scientific progress of the experiments for the back-end. If these parts are completely (almost completely) independent it is surely all more simple. In principles it is also possible a complete substitution of the back-end: both with a different program able to implement analysis of data of the AEgIS experiment or with a completely different program able to work with runs, to give outputs formatted in text and images, and to be configured by an xml file.

%2
\item
Another quite important point to keep in mind is that this application will be maintained and probably modified by people that aren't specialized web developers. So it is important to maintain the code as simple as possible and to "hide" the complexity where a complex structure is unavoidable. For example, the functions able to read and write on files, to extract histogram from vectors, to apply "regular expression" to verify the format of a file can be packed into functions able to be simply called without expect expertise from the programmer. All the decisions related to the paradigm and the environment to use for this project may take consideration about the fact that simplicity is a fundamental requirement.   

%3
\item
The logical separation between client and server implies that a way to communicate is needed. A simple communication based on Post and Get is quite effective, but a more structured communication based on exchange of complex object seems to be more re-usable, and simpler to maintain and modify. From this point of view Json language give a standard well-known solution, simple, efficient, and effective to solve this issue, but also other technologies were analysed: in particular we are looking for technologies able to send human-readable messages (this is useful in debug phase). The best alternative to Json is probably Yaml, that seems to be more readable, but also a more immature technology. In the following figure we can see the two different formats that encode the same message:

%TODO %TODO %TODO %TODO  INSERISCI UN IMMAGINE DI CONFRONTO TRA JSON E YAML

As we can see Yaml is very near to the way in which a human thinks, and has the advantage to be a superset of Json (so each Json message is also a Yaml message, but not viceversa) but this technology is less diffused and probably for the next programmers who will work on this application or will need to communicate through HTTP whit is having to work with a not-mainstream technology can create difficulties able to overcome advantages. 

%4
\item
The program of data analysis (gAn) need to be configured. Some of these configurations are accessible by the user, some aren't, but in both cases it is important to give a structured way to persist the configuration of the application. The first idea is to use a simple text file to achieve this goal, but with the growth of the file a more formal solution becomes preferable. Xml is a good solution, because it is a diffused standard and there are a lot of API and libraries able to help to write and read it easily.

%5
\item
A simple but precise way to evaluate the formatting of a string or a group of strings is needed to ensure in every moment that the program is working with the correct value. For example, it is important to ensure that if the user is inserting a run number the entered value is effectively a decimal number: in other case the front-end can advise the user of the error. This first example is related to an inadvertent error, but it is important to consider also other situations when, for security reason, immediately before save a number in a file, or immediately before insert a number as a variable in a bash script the application needs to ensure that the value is compatible (so, has the same format) of the expected one. This is a prudential approach, but we must consider the this application is thought to work on a Intra-net, not, on the internet, so the occurrence of an attack is not so immediate. There are some solution considered in order to avoid problems and formatting objects correctly, the two effectively used are regular expressions to certify if a string respect some limits, and some standardized functions provided by Php to check if in a string there are dangerous commands.
 
%6
All the software must work easily with the Root Framework, so communicate with it using APIs or other ways to interact with gAn. This is not always simply, because Root provides some APIs to communicate with languages used in web developing but there is a big lack of documentation. Often, in absence of documentation for some needed functionalities the best way to work was check directly the open source code of the APIs to understand the functionalities gave by them.

\end{enumerate}

\section{Why a web solution?}

Before start to work with this project using web development some other solution was evaluated:

\begin{enumerate}

% 1
\item 
The first solution considered was simply work with the commands of a common linux terminal. This is the solution implemented for gAn. This solution is surely powerful and effective, because gives the maximum control on the possible commands, allows pipelines, gives an immediate feedback in every moment (these features of the linux terminal are often ignored). However this is not the best possible available solution, because it presupposes a perfect understanding of the behavior of the program, and a very good memory to remember all the commands and all the possible analyzes. 
This solution is no longer used for the final application itself, but it is still very useful to debug the application in case of errors.   

% 2
\item
Another analyzed solution was the creation of a standalone application, with an advanced graphical interface, able to perform analyses on a common laptop or on the computers of the AEgIS control room. This solution is able to use all the advantages of the application of the human machine interaction science, but it is quite limited: in a hypothetical situation with this solution applied it would be very difficult give to all the users the same version of the application (that is changing continuously). So in that case a system able to check automatically the application for updates every "x" days would be necessary, and the total amount of complexity would be quite high. Other problems with this solution are that it would be able to work only with the LAN of the AEgIS Control Room to access the data that are stored on dedicated hard disks, and the computational power of a common laptop would have some problem to analyze data in a reasonable time (the data at the moment are some terabytes, but in the future this amount will probably growth rapidly).

%3
\item
Probably the better possible solution (the adopted one) is the web application. So with simply a web browser also a low quality laptop can access information, and both the data and the application are always at the last version (the application change quite rapidly, but actually new data are added every some hundreds of seconds every days when the machine is working.. so only a centralized application can meet the need to react to the continuous modification of the situation).
In this way the user doesn't need to install nothing, and this is an advantage because the practice shows that the installation process of gAn is not so linear and shows often configuration problems (the behavior of gAn is linked to the version of Root Framework installed on the machine, the different versions are not perfectly compatible). Another advantage of this solution is that in this way only the server needs to manage the problem to recover the data (at their most recent version) and to store them in an intelligent way. Furthermore the computational power of the server is probably more suitable to execute the requested analysis. It is important to reason about the possible request of parallel execution from more users at the same time, but considering the number of the total user of this application, that is around at maximum a few tens, the charge on the server to satisfy the multiple requests is not so high.    

\end{enumerate}

\section{Different hypothesis about how to implement the project}

After being sure about the choice of implementing a web solution it is important, considered the requirements take some decisions about the way in which work.
For the client side the situation seems to be clear: a HTML5 and Javascript solution, together with some kind of CSS processing technology, can meet all that is needed.

For the server side the challenge is a bit more open: there are to important decision to take: which programming language is more suitable for the exigences and which paradigm to adopt. This two question are in this case very linked together, and there are three solutions each one with pros and cons:

\begin{enumerate}

% 1: procedural PHP
\item
The first solution is simple: using Php with a Procedural approach. Php is born to be a procedural language able to work in the server side, and this approach can guarantee the simpler way to meet our goal. In particular this solution will allow also people that are not specialized in web developing to work easily on this project. On the other hand this solution is not very structured, and in a big project, specially with a numerous group of people that needs to work together will become full of problems.

% 2: OO PHP
\item
An alternative solution is again Php, but adopting the Object Oriented paradigm. This solution is probably more suitable for big server sides, more easy to maintain and extend. But Php is not born as OO language, it supports the concept of object only from the 5th version, and a native OO language can be a better solution. This fact take us to the third choice:

% 3: OO JAVA
\item
Java is a natively Object Oriented language, and Jave Enterprise Edition is perfect to work on an application server like apache. This is a very structured solution, and we can use also very good frameworks like Spring to create a very well organized server side, scalable also for big projects. But there are also bad sides: this kind of application is more complex, needs specialized programmers to be modified and is not very good for little server sides. 

According to our exigences, this server side is not so big, the login of the application is quite similar to a procedure, and mostly of the developer in the AEgIS experiment come from experiences with C, so are used to work with the procedural approach. At this point the decision is to work with Php, adopting a procedural approach.  

\end{enumerate}
 
\section{The simple solution: MVC pattern}

The application of the Model View Controller pattern is usually a good idea in Web Development, specially in an application that needs to be rapidly modified and needs a special attention to the maintenance. This pattern is perfect when it works with an object oriented project, but we can implement its logic also with a procedural approach.

We can start with a precise definition of what a MVC pattern is; according to Wikipedia:

\begin{quote} 

Model–view–controller (MVC) is a software design pattern for implementing user interfaces on computers. It divides a given application into three interconnected parts in order to separate internal representations of information from the ways that information is presented to and accepted from the user. The MVC design pattern decouples these major components allowing for efficient code reuse and parallel development.

\end{quote}

The following figure can make the point more clear:

\begin{figure}[H]
\centering
\includegraphics[scale=0.25]{MVC.png}  
\caption{The general structure of the MVC pattern}
\end{figure}    

So the goal is to divide the application in "independent" modules:

\begin{enumerate}

%1, VIEW
\item The view is the surface of the system. In this project it is created with Html, Css, Javascript (with some parts of Php, used to create through the "echo" command part of Html code that are repetitive, for example lists ). The design of the view in this kind of projects is particularly important because the application needs to represent particular kinds of information to particular kinds of users, and both the importance and the special feature needed by this interface are explained in the previous chapters. In the MVC pattern the view is the part less able to "reason" its goal is mainly related to the representation of information. In this project some kinds of representations are quite complex, but this complexity is managed by other tiers, and the view needs only to show the users images, html constructions, and artefacts to access graphics. So the view is the client side? No, it isn't: in the client side there are also a part of the controller, able to interpret the inputs of the user and send them to the logic of the application.

%2, MODEL
\item In the MVC the model is the central module of the pattern. It contains the application's behavior and must be independent of the user interface. Its primary goal is managing the data of the application. In gAn Web it is a complex situation: excluding some little data related to the configuration and other minor form of persistence, the main data related to this application are the output of another application (gAn). If we consider the whole system the model of the composed application is represented by the raw root files taken in input by the system. Their loading, interpretation, analysis and activities related to the update of the view are the main activities of the model tier.   

%3, CONTROLLER
\item The main goal of a controller are related to receiving the command of the users, the interpretation of these commands, and the consequent actions to change the state of the model. In gAn Web the controller is represented by the javascript functions able to check, interpret, and send to the server the desires of the users, and the Php scripts able to receive this commands, start the cycle of analysis of the program, and work directly with the model 


\end{enumerate}

In this explanation of the theory the MVC seems to be perfect, but in the application some further consideration are needed.

\section{The alternative solution: MVVM pattern}

In this situation it is perfectly clear what is the view and what it needs to do, and also it is quite clear what functions are surely related to the model. The definition of controller seems to be more nuanced: it is clear that this tier needs to receive the input of the user (and to manage it, that in some cases is not so simple), give them to the model that implements the business logic of the application and send to the view the results, so its role of "glue" is clear, but during the practical implementation an observation has grown: actually some functionalities that in the theory belongs to the controller, such as managing the complex interaction of the users when it works with the "alive" images provided by root and give him back a real-time feedback on the screen are in practice considerable part of the "logic" of the presentation, and furthermore their logic is implemented in the same methods able to read informations from the data so in some cases the strong division between view and controller seems to be less strong. 
In practice in this case a "fat view" manage the data-binding talking with an extension of the model, because this is the smartest way. 
At this point it is important to consider if in this project another way can be followed: apply a variation of the MVC: the Model View ViewModel pattern. 
In this pattern instead of the controller there is a component named ViewModel that allows the view to implement a logic, for example in this case interpreting the desires of the users and calling directly model's methods, and giving back to the view an image, modified in real time, in svg format. 

This pattern is less mainstream, but in this case can provide a more direct way to manage complex user interactions, at least in the page related to the output of the program, where the interaction user-data though images is quite complex. 
However it seems to be not perfectly suitable for all this project, and an intermediate solution can be adopt it only in little parts, where it can really be useful, in derogation of the MVC, that is the main pattern of this project.


 
\section{Other Used Technologies}

The adopted technologies for this project are generically the standard ones of a web application project, but some of them deserve some little explanation:

\begin{enumerate}

%1 sass
\item
SASS:
\\
\noindent
Syntactically Awesome Style Sheets is a language that aims to simplify the tasks related to the CSS files. The goals of this solution are not related directly with the functionalities of the application but more with the aesthetic of it.
The normal CSS language is simple and well organized, but using Sass we can further improve its features:

According to its official documentation :
\begin{quote}
Sass is an extension of CSS that adds power and elegance to the basic language. It allows you to use variables, nested rules, mixins, inline imports, and more, all with a fully CSS-compatible syntax. Sass helps keep large style sheets well-organized, and get small style sheets up and running quickly, particularly with the help of the Compass style library.

\end{quote}

In other words we can use Scss to create files similar to the normal CSS files, but with some improvements, such as nesting based inheritance and compile automatically them in normal CSS files, that are understandable by all the modern browser.

Some interesting features of Scss that can help this project are mainly two:
First of all the fact that Scss admits the concept of variable, so is more similar to a programming language that CSS; Second, probably more interesting, the existence of the nesting inheritance, in which the classes that are included in other classes full inherits the parent's features. This structure allows to produce elegant, better organizer, more reusable CSS code.

\item
Bootstrap:
\\
\noindent
This framework is mainly aimed to create a responsive (partially adaptive), well organized front-end. It is not a complete framework, so it not involves all the logical part of the application, but only the front-end of the web part. This fact is important because ensures us not to create conflicts with the Root Framework on the back-end, and if in a future the back-end part of the application will be strongly changed (it is quite likely), the existing strong separation will avoid without problem conflicts with this technology. Bootstrap is a not-invasive, light structure, that carries many advantages, but for this project there are in particular two Bootstrap's features that invite us to work with it: the availability of a responsive structure created with very simple commands and the existence of a lot of pre-produced standard graphic artifacts. 

The first one is quite useful for any web project, because Bootstrap allows to divide the screen in twelve columns and assign to each component a number of columns to occupy (recursively, each component can in turn be divided in twelve columns and so on). This functionality can link easily the dimension of the component to a screen size and redistribute the components in the screen automatically if the dimension of the screen changes. There are also more advanced structures that allow to easily tell the component to occupy a different number of columns or adopt different behaviors if the dimensions of the screen are particularly little, giving to Bootstrap an interesting role in the development of web applications for the mobile world, but in this case the application is thought to work on a laptop or on a pc, so the general behavior is sufficient to satisfy the requests.


% %TODO %TODO %TODO %TODO %TODO %TODO  caccia una foto di bootstrap con le colonne qui

The second is the availability of a big amount of already implemented graphic objects such as buttons, fields, widgets. This allows us not only to save time (implementation of the style sheet of a widget is easy.. the time saving would be actually little..) but mostly to re-use solution that was created by specialized personnel with graphic design experience, so thought to have the strongest possible affordance. Another interesting point is that Bootstrap is quite diffused on the web, many sites are created using Bootstrap's artifacts, so the user can re-use his experience to understand the meaning of some commands. 
  
\item
Json:
\\
\noindent
Json is a smart way to communicate between client and server.
It is an open format that uses human-readable text to transmit data objects consisting of attribute–value pairs. The interesting point of this technologies is that it is (quite) easy to read by a human. This feature is very useful in debugging phase.  
It is the most common data format used for asynchronous browser/server communication, largely replacing XML, and is used by AJAX. In this application each server module produces in output Json-formatted information, and the clients can (using javascript) interpret these information and use them to build the page, and communicate actively with the server sending him requests formatted using this standard.
The diffusion and language agnosticism of Json allows it to be a common languages among different applications, and makes the parts of gAn Web more independent: if we want for example to replace only the client side we can simply build a new client side able to communicate with Json without touching the server. 

\end{enumerate}  

%TODO %TODO %TODO %TODO %TODO %TODO  INSERT AN IMAGE WITH JSON TO EXPLAIN WHAT IS

\section{General Schema of the project}

In the following picture we try to explain the general structure of the project; This picture isn't a standard UML graph, because a more free and informal graph can in this case explain in a probably less precise but more simple way what are the roles of each part. In particular, in this picture there are both software and hardware components, because we want to clarify what is the generic flow of the system (not the flow of the software system, but the global flow of information in the AEgIS experiment). To meet the goal of explain what are the main actor in this situation some particulars are omitted, which the aim of exposing only what we want to clarify.

\begin{figure}[H]
\centering
\includegraphics[scale=0.25]{generalAEgISImage.png} 
\caption{The generic structure of the experiment}
\end{figure}
   
Through this image the general situation is more understandable: gAn Web is the whole client of the application, and occupy also a part of the server. The view is the front end of the client, the controller is the bridge between the view and the model. We observe that the controller is divided in two part: the one on the client and the one on the server, and this part are strongly separates and able to communicate with json (actually, this explanation of the situation of the controller is simplified, as exposed in the previous paragraphs related to MVC and MVVM). It is very easy if needed to exchange the client or the server with another, and this leads that whereas the concepts of view and model are clearly defined and quite simply the controller is quite smoked (de facto, in the implementation of the images page the adopted solution is a MVVM). DAQ is the system of hardware and software able to take information from the machine and put them into raw, unfiltered Root files. DAQ is a very complex and heterogeneous machine, its description is not part of the goal of this document, but its role of information transporter is a key-role of the whole system. Also the AEgIS' experiment machine is a complex system, partially described in the first chapter, its role is to be the physical source of the whole information, the beginning of the system.

      