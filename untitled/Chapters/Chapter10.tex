% Chapter 10

\chapter{Conclusions} % Main chapter title

\label{Chapter 10} % For referencing the chapter elsewhere, use \ref{Chapter1} 

%----------------------------------------------------------------------------------------

In this chapter some important points of this document are summarized, and some observations about the future of this application are expressed.

\section{Conclusions}
The first thing that we can observe is that the result of this project is a fusion between the studies of the needs and characteristics of the users and the ways in which this software can meet their needs, and an organized development of the product using web technologies. The most important needs that have been identified in this application are the flexibility, so the ability to be adapted continuously to a changing environment and the ability to understand the features needed by a particular class of users.


In this cyclical process of development, after much trying and re-trying, we can surely say that we have tested a lot of different ways to give the user the best possible product. The big modifications between different stages, both the ones derived from the pilot users, and the ones derived from the general users probably identified the most of the problems on this application, that now, with the final version are hopefully solved. At this point we can recognize in the adoption of a cyclical process of development the best choice made in this project, the most able to produce added-value to the result.

A good advantages that we have enjoyed during the development is surely related to the possibility to get hints from the pilot users, that are both users and supervisors, and surely gave a big help to correctly interpret the needs of the typical users. Also the availability of the general users to test the application and to help with ideas and opinions was surely important.

I conclusion we must observe that the needs of the users are not stable, are continuously changing, and the rapid evolution of the web technologies makes very unlikely to consider this application stable in the time. So, even if gAn Web is designed to be easy to modify, and in case to be enriched with new functionalities, the biggest challenge for the future is try to maintain the application up to date in an unstable and dynamic environment.  

