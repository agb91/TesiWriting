% Chapter 10

\chapter{Conclusions} % Main chapter title

\label{Chapter 10} % For referencing the chapter elsewhere, use \ref{Chapter1} 

%----------------------------------------------------------------------------------------

In this chapter some important points of this document are summarized, and some observations about the future of this application are expressed.

\section{Goals of this project}
The experiment AEgIS at CERN is aiming at measuring the gravitational interaction between matter and antimatter. The data are collected by a series of detectors, through dedicated electronics and a custom DAQ system. An Object Oriented C++ program, called gAn, has been developed to analyse such data to extract "meaningful" information. This program can be run on a linux terminal via command lines. The goal of this thesis was to develop a web interface that could interact with the "gAn" to make the usage more friendly and easier. We called this application "gAn Web". 

%%testo copiato
\section{Adopted strategy to achieve the goals}
The whole project is based on the study and the application of the principles of the Human Computer Interaction (HCI), in particular the ones proposed by Jakob Nielsen [6]. This science studies how the system composed by the users and the software he is using works, and all the phenomena around it. This approach was necessary in order to obtain the best possible solution for the particular kind of users of the AEgIS experiment and the particular domain of particles physics. 
The project is implemented as a web application, in order to avoid every problem related to the installation and to centralize all the inevitable updates and modifications towards that every project is directed. 

\section{Work carried on}

The first problem encountered was the deep difference between the domain of computer engineering (the domain of the developer) and the domain of physics. In order to fill at least partially this gap before the production of this software the developer worked for some weeks directly in the AEgIS experiment during the process of data taking: this experience was extremely important to understand how this domain works and what are the particular need of the users.

From the HCI point of view several ways were tried in order to produce a solution able to achieve the goals. 
The design process was user-centered and open to the direct participation of the user to the design itself: this means that before all there was a study of the effective needs of the users and about their peculiarities, the users were consulted for a wide range of decisions, and every solutions was tested directly with the users every time that it was possible. 
  
This project applied the state of art of HCI adopting an iterative, star-shaped model of development. In this design process the central and more important activity was the evaluation of usability, that was considered more important that the implementation of the project itself. The evaluations of usability aimed to understand if the current iteration was able to satisfy the needs of the users, and around this process all the other activities (functional analysis, specification of requirements, prototyping, implementation) were carried on, more than once, in order to produce and test three increasingly improving stages of the application. 
This kind of solution may seems slow and inefficient but the iterative nature of this project, who repeats more times almost all the activities leads us to a gradual approach to the right solution, producing iteratively versions continuously more able to achieve the goals.

These solutions were accurately tested directly with the users in a range of ways, from questionnaires and interviews to direct observation both in real and simulated work situations.
The tests took place both at the end of the design project and during it, because these weren't only acceptance tests aimed to check the quality of the application, but formative tests, needed to try different solutions during the design process in order to be sure to reach as best as is possible the need of the users.

From the technical point of view the choice to implement the software as web application can ensure flexibility, easy access for the users, centralization of the modifications. Moreover, the strong division between front-end and back-end, and  
the logical isolation of the code related to the analyses that is the part of code more subject to changes improve the modularity and the maintainability of the application.
The application was written in PHP, Javascript, HTML5, Sass. This solutions was considered more suitable than the other ones considered because PHP allows to implement easily the server-side, Javascript combined with CSS and HTML5 on the client side allows to implement an advanced GUI with a little effort, using also some 3D libraries freely available (Root Js, based on Three JS). The adoption of the framework Bootstrap improves the simplicity of the application making it responsive without implement complex commands, and gives us a lot of graphic features such as icons and widget very diffused on the internet, based on consolidated standards and produced by professional graphics. All the software was developed keeping in mind to maintain all the system as simple as possible, in this way in the future other developers will be able to maintain easily this application when needed.

\section{Final result}

The resulting application is an integration between DAQ, gAn and gAn web .This is a flexible solution able to allow the users with simply a web browser to analyse and extract complex information from the raw files produced by the experiment. The interface is simple and clear, it is thought to work correctly with the other software of the AEgIS experiment in the typical work situation of data taking. The range of analyses that this application can execute can vary very fast, in a transparent way for the users, because it is sufficient add the new analyses in the correct folder or delete the old ones to let the system able to work with them. The communication between client and server are in Json, so they are platform agnostic and ensure a complete independence between modifications in the client side and in the server side.

\section{Considerations and future developments} 
A good advantages that we have enjoyed during the development is surely related to the possibility to get hints from the pilot users, that are both users and supervisors, and surely gave a big help to correctly interpret the needs of the typical users. Also the availability
of the general users to test the application and to help with ideas and opinions was surely important.
In conclusion we must observe that the needs of the users are not stable, are continuously changing, and the rapid evolution of the web technologies makes very unlikely to consider this application stable in the time. So, even if gAn Web is designed to be easy to modify, and in case to be enriched with new functionalities, the biggest challenge for the future is try to maintain the application up to date in an unstable and dynamic environment.